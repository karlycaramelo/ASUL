\documentclass[a4paper, 11pt, oneside]{article}

\newcommand{\plogo}{\fbox{$\mathcal{PL}$}} 
\usepackage{amsmath}
\usepackage[utf8]{inputenc} 
\usepackage[T1]{fontenc} 
\usepackage{enumitem}
\usepackage{graphicx}
\usepackage{graphicx}
\usepackage{supertabular}
\usepackage[spanish]{babel}
\graphicspath{{Imagenes/}}

\begin{document} 

\begin{titlepage} 

	\centering 
	
	\scshape 
	
	\vspace*{\baselineskip} 
	
	
	
	\rule{\textwidth}{1.6pt}\vspace*{-\baselineskip}\vspace*{2pt} 
	\rule{\textwidth}{0.4pt} 
	
	\vspace{0.75\baselineskip} 
	
	{\LARGE System Kickstart}	
	\vspace{0.75\baselineskip} 
	
	\rule{\textwidth}{0.4pt}\vspace*{-\baselineskip}\vspace{3.2pt}
	\rule{\textwidth}{1.6pt} 
	
	\vspace{2\baselineskip} 
	

	ADMINISTRACIÓN DE SISTEMAS UNIX/LINUX
	
	\vspace*{3\baselineskip} 
	
	
	
	Alumna:
	
	\vspace{0.5\baselineskip} 
	
	{\scshape\Large Karla Adriana Esquivel Guzmán \\} 
	\vspace{0.5\baselineskip} 
	\vfill
	\includegraphics{unam.jpg}
	
	\textit{UNIVERSIDAD NACIONAL AUTONOMA DE MEXICO} 
	
	\vfill
	
	
	
	
	\vspace{0.3\baselineskip} 
	
	03/Mayo/2019 
	
	

\end{titlepage}
Kickstart puede proporcionar una manera relativamente fácil de realizar una implementación masiva y puede ser totalmente automatizado. Es fácil tener varias configuraciones Kickstart diferentes listas para usar, cada una con una configuración de implementación diferente.\\
Los archivos de configuración de Kickstart pueden ser construidos de tres maneras:
\begin{itemize}
    \item A mano.
    \item Usando la herramienta con interfaz gráfica system-config-kickstart.
    \item Usando Anaconda, el programa de instalación estándar de Red Hat.
    \item Anaconda producirá el archivo de configuración anaconda-ks.cfg al final de cualquier instalación manual. Este archivo puede ser usado para reproducir automáticamente la misma instalación o editarlo (manualmente o con system-config-kickstart).

\end{itemize}


\end{document}
