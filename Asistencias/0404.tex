\documentclass[a4paper, 11pt, oneside]{article}

\newcommand{\plogo}{\fbox{$\mathcal{PL}$}} 
\usepackage{amsmath}
\usepackage[utf8]{inputenc} 
\usepackage[T1]{fontenc} 
\usepackage{enumitem}
\usepackage{graphicx}
\usepackage{graphicx}
\usepackage{supertabular}
\usepackage[spanish]{babel}
\graphicspath{{Imagenes/}}

\begin{document} 

\begin{titlepage} 

	\centering 
	
	\scshape 
	
	\vspace*{\baselineskip} 
	
	
	
	\rule{\textwidth}{1.6pt}\vspace*{-\baselineskip}\vspace*{2pt} 
	\rule{\textwidth}{0.4pt} 
	
	\vspace{0.75\baselineskip} 
	
	{\LARGE IPTables}	
	\vspace{0.75\baselineskip} 
	
	\rule{\textwidth}{0.4pt}\vspace*{-\baselineskip}\vspace{3.2pt}
	\rule{\textwidth}{1.6pt} 
	
	\vspace{2\baselineskip} 
	

	ADMINISTRACIÓN DE SISTEMAS UNIX/LINUX
	
	\vspace*{3\baselineskip} 
	
	
	
	Alumna:
	
	\vspace{0.5\baselineskip} 
	
	{\scshape\Large Karla Adriana Esquivel Guzmán \\} 
	\vspace{0.5\baselineskip} 
	\vfill
	\includegraphics{unam.jpg}
	
	\textit{UNIVERSIDAD NACIONAL AUTONOMA DE MEXICO} 
	
	\vfill
	
	
	
	
	\vspace{0.3\baselineskip} 
	
	04/Abril/2019 
	
	

\end{titlepage}
Esta clase visitamos el sitio web Distrowatch, en el cual se encuentra información diversa sobre linux y sus distribuciones así como un ranking de las distribuciones mas utilizadas. Y vimos que en los últimos 6 meses las primeras 10 distribuciones más utilizadas son:
\begin{itemize}

    \item	MX Linux	
    \item	Manjaro	
    \item	Mint	
    \item	elementary	
    \item	Ubuntu	
    \item	Debian	
    \item	Fedora	
    \item	Solus	
    \item	openSUSE
    \item	deepin
\end{itemize}
\section*{Extra}
\textbf{Más sobre Distrowatch}\\
El sitio fue iniciado el 31 de mayo de 2001, y desde entonces lo mantiene Ladislav Bodnar. En sus inicios, solamente contaba con una tabla muy simple con las cinco Distribuciones Linux más grandes en aquel momento, donde se comparaban varias características (precio, versión, fecha de realización, Con el tiempo, el sitio ha aumentado en contenido y especificación de éste. El crecimiento en desarrollo de las distribuciones GNU/Linux, BSD, y Solaris, y su uso masivo y diversificación, ha permitido mantener una gran base de datos relacionados a las continuas modificaciones, actualizaciones y discontinuidades del software libre y de código abierto en general.
\end{document}
