\documentclass[a4paper, 11pt, oneside]{article}

\newcommand{\plogo}{\fbox{$\mathcal{PL}$}} 
\usepackage{amsmath}
\usepackage[utf8]{inputenc} 
\usepackage[T1]{fontenc} 
\usepackage{enumitem}
\usepackage{graphicx}
\usepackage{graphicx}
\usepackage{supertabular}
\usepackage[spanish]{babel}
\graphicspath{{Imagenes/}}

\begin{document} 

\begin{titlepage} 

	\centering 
	
	\scshape 
	
	\vspace*{\baselineskip} 
	
	
	
	\rule{\textwidth}{1.6pt}\vspace*{-\baselineskip}\vspace*{2pt} 
	\rule{\textwidth}{0.4pt} 
	
	\vspace{0.75\baselineskip} 
	
	{\LARGE Modelo OSI (Open System Interconnection)}	
	\vspace{0.75\baselineskip} 
	
	\rule{\textwidth}{0.4pt}\vspace*{-\baselineskip}\vspace{3.2pt}
	\rule{\textwidth}{1.6pt} 
	
	\vspace{2\baselineskip} 
	

	ADMINISTRACIÓN DE SISTEMAS UNIX/LINUX
	
	\vspace*{3\baselineskip} 
	
	
	
	Alumna:
	
	\vspace{0.5\baselineskip} 
	
	{\scshape\Large Karla Adriana Esquivel Guzmán \\} 
	\vspace{0.5\baselineskip} 
	\vfill
	\includegraphics{unam.jpg}
	
	\textit{UNIVERSIDAD NACIONAL AUTONOMA DE MEXICO} 
	
	\vfill
	
	
	
	
	\vspace{0.3\baselineskip} 
	
	21/Marzo/2019 
	
	 

\end{titlepage}
El modelo de interconexión de sistemas abiertos, también llamado OSI (en inglés open system interconnection) es el modelo de red descriptivo propuesto por la Organización Internacional para la Estandarización (ISO) en el año 1977 y aprobado en el año 1984. Es una normativa formada por siete capas que define las diferentes fases por las que deben pasar los datos para viajar de un dispositivo a otro sobre una red de comunicaciones. Constituye por tanto un marco de referencia para la definición de arquitecturas de interconexión de sistemas de comunicaciones. En este estándar no se define una implementación de una arquitectura de red, sino que se establece un modelo sobre el cual comparar otras arquitecturas y protocolos. El modelo OSI establece una arquitectura jerárquica estructurada en 7 capas. La idea es descomponer el proceso complejo de la comunicación en varios problemas más sencillos y asignar dichos problemas a las distintas capas, de forma que una capa no tenga que preocuparse por lo que hacen las demás. Según la estructura jerárquica, cada capa realiza servicios para la capa inmediatamente superior, a la que devuelve los resultados obtenidos, y a su vez demanda servicios a la capa inmediatamente inferior.

\section*{Extra}
\textbf{Algo más sobre OSI}\\
El modelo de referencia OSI es el modelo principal para las comunicaciones por red. Aunque existen otros
modelos, en la actualidad la mayoría de los fabricantes de redes relacionan sus productos con el modelo de referencia OSI, especialmente cuando desean enseñar a los usuarios cómo utilizar sus productos. Los
fabricantes consideran que es la mejor herramienta disponible para enseñar cómo enviar y recibir datos a través de una red. El modelo de referencia OSI permite que los usuarios vean las funciones de red que se producen en cada capa. Más importante aún, el modelo de referencia OSI es un marco que se puede utilizar para comprender cómo viaja la información a través de una red. Además, puede usar el modelo de referencia OSI para visualizar cómo la información o los paquetes de datos viajan desde los programas de aplicación a través de un medio de red (por ej., cables, etc.), hasta otro programa de aplicación ubicado en otro computador de la red, aún cuando el transmisor y el receptor tengan distintos tipos de medios de red. 
\end{document}
