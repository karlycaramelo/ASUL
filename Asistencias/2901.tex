\documentclass[a4paper, 11pt, oneside]{article}

\newcommand{\plogo}{\fbox{$\mathcal{PL}$}} 
\usepackage{amsmath}
\usepackage[utf8]{inputenc} 
\usepackage[T1]{fontenc} 
\usepackage{enumitem}
\usepackage{graphicx}
\usepackage{graphicx}
\usepackage{supertabular}
\usepackage[spanish]{babel}
\graphicspath{{Imagenes/}}

\begin{document} 

\begin{titlepage} 

	\centering 
	
	\scshape 
	
	\vspace*{\baselineskip} 
	
	
	
	\rule{\textwidth}{1.6pt}\vspace*{-\baselineskip}\vspace*{2pt} 
	\rule{\textwidth}{0.4pt} 
	
	\vspace{0.75\baselineskip} 
	
	{\LARGE Resumen 02: Primera Ayudantía}	
	\vspace{0.75\baselineskip} 
	
	\rule{\textwidth}{0.4pt}\vspace*{-\baselineskip}\vspace{3.2pt}
	\rule{\textwidth}{1.6pt} 
	
	\vspace{2\baselineskip} 
	

	ADMINISTRACIÓN DE SISTEMAS UNIX/LINUX
	
	\vspace*{3\baselineskip} 
	
	
	
	Alumna:
	
	\vspace{0.5\baselineskip} 
	
	{\scshape\Large Karla Adriana Esquivel Guzmán \\} 
	\vspace{0.5\baselineskip} 
	\vfill
	\includegraphics{unam.jpg}
	
	\textit{UNIVERSIDAD NACIONAL AUTONOMA DE MEXICO} 
	
	\vfill
	
	
	
	
	\vspace{0.3\baselineskip} 
	
	29/Enero/2019 
	
	 

\end{titlepage}

La segunda clase tocó con el Ayudante, principalmente hablamos de comandos básicos de Linux y hubo una breve una introducción al concepto \textbf{Variable de entorno}, una variable de entorno es un valor dinámico cargado en la memoria, que puede ser utilizado por varios procesos que funcionan simultáneamente. También se habló sobre el concepto de \textbf{Variable Global}, una variable global es visible y valida en cualquier script de Bash, cualquier variable que se declare en bash por default es una variable global. A continuación enlistaré también los comandos vistos en clase. 

\begin{itemize}
 \item echo \$PATH: Sirve para ver las ubicaciones de los comandos a ejecutar.
 \item man(Manual): Sirve para ver información sobre los comandos.
 \item env: Si env se ejecuta sin ninguna opción, muestra las variables del entorno actual. De lo contrario, env establece cada nombre para un valor y ejecuta el comando.
 \item printenv: Imprime la(s) variable(s) de entorno específicadas en caso de no ser estar especificada dicha(s) variable(s), entonces se imprimen pares y valores de todas las variables de entorno.
 \item grep(global regular expression print): procesa el texto línea por línea e imprime cualquier línea que coincida con un patrón específico.
 \item cat: sirve para visualizar archivos de texto, combinar copias de ellos y crear nuevos.
 \item source: se puede usar para cargar cualquier archivo de funciones en el script de shell actual o en un prompt.
 \item ps: Sirve para ver los procesos corriendo en el sistema.
 \item top: Permite a los usuarios monitorear los procesos y el uso de recursos del sistema en Linux.
 \item ls -la: Sirve para ver archivos ocultos.
 \item nano .bashrc: nano es un editor de texto que sirve para abrir .bashrc, 
el archivo .bashrc es un script de shell que se ejecuta cada vez que un usuario abre un nuevo shell.
\end{itemize}
\newpage
Cuando se utiliza el comando \textbf{ps} aparecen las siglas  PID TTY(como se muestra en la siguiente imagen), el ayudante pidió que se investigara cuál es el significado de estas.

\begin{center}
 \includegraphics[scale=0.30]{1.png}
\end{center}

\begin{tabular}{ll}
PID &- Es el número de Procesos\\
TTY &- Es el nombre de la consola en la que el usuario\\ &ha iniciado sesión.
\end{tabular}

Se mencionó que los estados de un proceso son dos:
\begin{itemize}
 \item Ready
 \item Pause
\end{itemize}

por último se mostró en clase la estructura de como definir una función en bash y se hace de la siguiente manera:
\begin{verbatim}
                    Nombre_Función(){
                      < Comandos >
                    }
\end{verbatim}

\end{document}
