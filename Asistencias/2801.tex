\documentclass[a4paper, 11pt, oneside]{article}

\newcommand{\plogo}{\fbox{$\mathcal{PL}$}} 
\usepackage{amsmath}
\usepackage[utf8]{inputenc} 
\usepackage[T1]{fontenc} 
\usepackage{enumitem}
\usepackage{graphicx}
\usepackage{graphicx}
\graphicspath{{Imagenes/}}

\begin{document} 

\begin{titlepage} 

	\centering 
	
	\scshape 
	
	\vspace*{\baselineskip} 
	
	
	
	\rule{\textwidth}{1.6pt}\vspace*{-\baselineskip}\vspace*{2pt} 
	\rule{\textwidth}{0.4pt} 
	
	\vspace{0.75\baselineskip} 
	
	{\LARGE Resumen 01: Introducción al curso}	
	\vspace{0.75\baselineskip} 
	
	\rule{\textwidth}{0.4pt}\vspace*{-\baselineskip}\vspace{3.2pt}
	\rule{\textwidth}{1.6pt} 
	
	\vspace{2\baselineskip} 
	

	ADMINISTRACIÓN DE SISTEMAS UNIX/LINUX
	
	\vspace*{3\baselineskip} 
	
	
	
	Alumna:
	
	\vspace{0.5\baselineskip} 
	
	{\scshape\Large Karla Adriana Esquivel Guzmán \\} 
	\vspace{0.5\baselineskip} 
	\vfill
	\includegraphics{unam.jpg}
	
	\textit{UNIVERSIDAD NACIONAL AUTONOMA DE MEXICO} 
	
	\vfill
	
	
	
	
	\vspace{0.3\baselineskip} 
	
	28/Enero/2019 
	
	 

\end{titlepage}

El primer día de clase se dio una breve introducción sobre los temas que se verán en el curso, mencionando que el temario es muy extenso y dependerá del avance del grupo hasta donde alcanzaremos a llegar a en el temario. Principalmente se preguntó a cada uno de los alumnos su expectativa sobre el curso, en el siguiente listado mostraré los diversos intereses y expectativas que tengo sobre el curso.

\begin{itemize}
 \item Principalmente espero aprender más sobre seguridad y perfeccionar mi conocimiento en el manejo de sistemas Linux, para poder explotar al máximo su utilidad.
 \item Mejorar mi conocimiento sobre redes y sus diversas aplicaciones.
 \item Conocer más sobre sistemas de archivos.
 \item Aprender a programar en Bash y más información sobre AWK puesto que estoy muy poco familiarizada con este lenguaje de programación.
 \item Al finalizar el curso espero poder solucionar problemas que se me presenten con Linux con mayor facilidad y me gustaría poder entender aún mejor porque ocurren estos errores.
\end{itemize}

Se habló también sobre la evaluación del curso y las condiciones para poder aprobar el curso.

\begin{enumerate}
 \item La condición principal es tener un 80\% de asistencia para tener derecho a calificación aprobatoria en el curso(asistencia tanto con él ayudante como con el profesor).
 \item Los porcentajes de evaluación son los siguientes:
 \begin{itemize}
  \item 30\% Examenes
  \item 40\% Prácticas
  \item 10\% Tareas 
  \item 10\% Participaciones
  \item 10\% Exposiciones
 \end{itemize}
\end{enumerate}

Además se definieron las fechas de examenes las cuales pondré en un listado a continuación:
\begin{align*}
 Primer Parcial &- 15/02/2019\\
 Segundo Parcial &- 15/03/2019\\
 Tercer Parcial &- 29/03/2019\\
 Cuarto Parcial &- 26/04/2019\\
 Quinto Parcial &- 17/05/2019
\end{align*}

La primera tarea que se dejó en el curso fue crear un repositorio llamado \textbf{ASUL} en \textbf{GitHub} y mandarlo por correo al Profesor con copial al Ayudante.






\end{document}

