\documentclass[a4paper, 11pt, oneside]{article}

\newcommand{\plogo}{\fbox{$\mathcal{PL}$}} 
\usepackage{amsmath}
\usepackage[utf8]{inputenc} 
\usepackage[T1]{fontenc} 
\usepackage{enumitem}
\usepackage{graphicx}
\usepackage{graphicx}
\usepackage{supertabular}
\usepackage[spanish]{babel}
\graphicspath{{Imagenes/}}

\begin{document} 

\begin{titlepage} 

	\centering 
	
	\scshape 
	
	\vspace*{\baselineskip} 
	
	
	
	\rule{\textwidth}{1.6pt}\vspace*{-\baselineskip}\vspace*{2pt} 
	\rule{\textwidth}{0.4pt} 
	
	\vspace{0.75\baselineskip} 
	
	{\LARGE Resumen 09: Servicios de Conexión Remota}	
	\vspace{0.75\baselineskip} 
	
	\rule{\textwidth}{0.4pt}\vspace*{-\baselineskip}\vspace{3.2pt}
	\rule{\textwidth}{1.6pt} 
	
	\vspace{2\baselineskip} 
	

	ADMINISTRACIÓN DE SISTEMAS UNIX/LINUX
	
	\vspace*{3\baselineskip} 
	
	
	
	Alumna:
	
	\vspace{0.5\baselineskip} 
	
	{\scshape\Large Karla Adriana Esquivel Guzmán \\} 
	\vspace{0.5\baselineskip} 
	\vfill
	\includegraphics{unam.jpg}
	
	\textit{UNIVERSIDAD NACIONAL AUTONOMA DE MEXICO} 
	
	\vfill
	
	
	
	
	\vspace{0.3\baselineskip} 
	
	07/Febrero/2019 
	
	 

\end{titlepage}

Hoy en clase se habló sobre servicios de conexión remota los mencionados y más conocidos son:
\begin{itemize}
 \item TeamViewer
 \item TigerVNC Viewer
 \item Skype
 \item Hangouts
 \item Chrome Remote Desktop
\end{itemize}

Todos estos servicios te permiten desde visualizar un escritorio de otra computadora distinta a la que trabajas, o compartir tu pantalla por medio de una dirección IP, también por medio de estos servicios puedes conectarte a otra computadora y manejar el sistema como si estuvieras físicamente frente a ella. En clase con el profesor utilizamos TigerVNC Viewer para poder ver lo que hacía desde la terminal de la computadora que él estaba ocupando.

\section*{Más conceptos mencionados durante la clase:}
\begin{itemize}
 \item Se mencionó el comando \textbf{file} el cual sirve para mostrar información sobre el tipo de archivo.
 \item Se mencionó nuevamente el concepto \textbf{Archivo de texto plano}, es un simple archivo de texto formado por caracteres.
 \item Se mencionó nuevamente el concepto \textbf{Archivo Binario}, éste contiene información codificada en binario.
\end{itemize}

Durante la clase de laboratorio únicamente se dejó una practica de procesamiento de texto, en la cual teníamos que checar archivos de texto plano y manipularlos por medio de funciones en bash, utilizamos el directorio \textbf{/var/log/}. En el caso de mi equipo los archivos con los que trabajamos fueron:

\begin{itemize}
 \item ./apt/history.log
 \item ./fontconfig.log
 \item ./auth.log
 \item ./alternatives.log
\end{itemize}


\end{document}
