\documentclass[a4paper, 11pt, oneside]{article}

\newcommand{\plogo}{\fbox{$\mathcal{PL}$}} 
\usepackage{amsmath}
\usepackage[utf8]{inputenc} 
\usepackage[T1]{fontenc} 
\usepackage{enumitem}
\usepackage{graphicx}
\usepackage{graphicx}
\usepackage{supertabular}
\usepackage[spanish]{babel}
\graphicspath{{Imagenes/}}

\begin{document} 

\begin{titlepage} 

	\centering 
	
	\scshape 
	
	\vspace*{\baselineskip} 
	
	
	
	\rule{\textwidth}{1.6pt}\vspace*{-\baselineskip}\vspace*{2pt} 
	\rule{\textwidth}{0.4pt} 
	
	\vspace{0.75\baselineskip} 
	
	{\LARGE Resumen 07: Primera Clase}	
	\vspace{0.75\baselineskip} 
	
	\rule{\textwidth}{0.4pt}\vspace*{-\baselineskip}\vspace{3.2pt}
	\rule{\textwidth}{1.6pt} 
	
	\vspace{2\baselineskip} 
	

	ADMINISTRACIÓN DE SISTEMAS UNIX/LINUX
	
	\vspace*{3\baselineskip} 
	
	
	
	Alumna:
	
	\vspace{0.5\baselineskip} 
	
	{\scshape\Large Karla Adriana Esquivel Guzmán \\} 
	\vspace{0.5\baselineskip} 
	\vfill
	\includegraphics{unam.jpg}
	
	\textit{UNIVERSIDAD NACIONAL AUTONOMA DE MEXICO} 
	
	\vfill
	
	
	
	
	\vspace{0.3\baselineskip} 
	
	06/Febrero/2019 
	
	 

\end{titlepage}
Esta clase hubo un repaso de los comandos vistos el día anterior en clase a continuación voy a enunciarlos:
\begin{itemize}
 \item tail: La salida de tail son las últimas lineas de texto de algún archivo, por default devuelve las últimas 10 lineas.
 \item wc: Imprime un byte, la palabra, el contador de lineas, el contador de bytes, espacios en blanco separados de las palabras y nuevas lineas en cada archivo dado.
 \item grep: Utiliza una expresión regular e imprime las lineas de texto que coincidan con la expresión regular.
 \item cat: concatena archivos y los imprime.
 \item head: La salida de head por default son las primeras 10 lineas de código.
 \item sed: Sirve para modificar cada linea en un archivo, se utiliza para reemplazar una parte específica de la linea.
\end{itemize}

También se comentó que estos comandos sirven para procesar y              manipular cadenas de texto. Vimos ejemplos de como funcionaban y sólo eso.




\end{document}
