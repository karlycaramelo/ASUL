\documentclass[a4paper, 11pt, oneside]{article}

\newcommand{\plogo}{\fbox{$\mathcal{PL}$}} 
\usepackage{amsmath}
\usepackage[utf8]{inputenc} 
\usepackage[T1]{fontenc} 
\usepackage{enumitem}
\usepackage{graphicx}
\usepackage{graphicx}
\usepackage{supertabular}
\usepackage[spanish]{babel}
\graphicspath{{Imagenes/}}

\begin{document} 

\begin{titlepage} 

	\centering 
	
	\scshape 
	
	\vspace*{\baselineskip} 
	
	
	
	\rule{\textwidth}{1.6pt}\vspace*{-\baselineskip}\vspace*{2pt} 
	\rule{\textwidth}{0.4pt} 
	
	\vspace{0.75\baselineskip} 
	
	{\LARGE Compresores, Descompresores de Archivos y Routers}	
	\vspace{0.75\baselineskip} 
	
	\rule{\textwidth}{0.4pt}\vspace*{-\baselineskip}\vspace{3.2pt}
	\rule{\textwidth}{1.6pt} 
	
	\vspace{2\baselineskip} 
	

	ADMINISTRACIÓN DE SISTEMAS UNIX/LINUX
	
	\vspace*{3\baselineskip} 
	
	
	
	Alumna:
	
	\vspace{0.5\baselineskip} 
	
	{\scshape\Large Karla Adriana Esquivel Guzmán \\} 
	\vspace{0.5\baselineskip} 
	\vfill
	\includegraphics{unam.jpg}
	
	\textit{UNIVERSIDAD NACIONAL AUTONOMA DE MEXICO} 
	
	\vfill
	
	
	
	
	\vspace{0.3\baselineskip} 
	
	07/Marzo/2019 
	
	 

\end{titlepage}

En ésta clase se hablo sobre los Compresores y descompresores de archivos así como el trabajo que hacen los Routers.

\section*{Compresores y Descompresores de Archivos:}
Un compresor de archivos es un programa que permite reducir (comprimir) el tamaño de un archivo. Ésto se consigue mediante una serie de algoritmos que permiten que los datos contenidos en un archivo ocupen menos tamaño sin que se produzca perdida de información. Cuando hablamos de un compresor de archivos entendemos que éste incluye la capacidad de descomprimir archivos comprimidos. Por lo que a la definición anterior hay que añadir que el compresor también es capaz de restituir el archivo comprimido a su formato original (descomprimir). No podemos dejar comprimido el archivo pues el problema radica en que si comprimimos un archivo cambia su formato, su estructura, y solo puede ser manejado por los compresores. Por ejemplo si tienes un documento de texto con extensión PDF que puedes abrir con el programa Foxit PDF Reader y lo comprimes con un compresor de archivos ya no podrás abrir el documento hasta que lo descomprimas. Esto último pasa porque cuando comprimes un archivo cambia su extensión. Me explico, por ejemplo en un archivo llamado “mi\_programa.exe” la extensión es “.exe” y cuando lo comprimas cambiará a “mi\_programa.zip” o a “mi\_programa.rar” dependiendo del formato de compresión que elijas. Los formatos de compresión más habituales son estos dos (zip y rar), pero existen muchos otros como: 7-ZIP, A, ACE, ARC, ARJ, B64, BH, BIN, BZ2, BZA, C2D, CAB, CDI, CPIO, DEB, ENC, GCA, GZ, GZA, HA, IMG, ISO, JAR, LHA, LIB, LZH, MDF, MBF, MIM, NRG, PAK, PDI, PK3, RPM, TAR, TAZ, TBZ, TGZ, TZ, UUE, WAR, XXE, YZ1, Z y ZOO.

La mayoría de compresores te permitirán manejar cualquiera de estos formatos aunque si el sistema operativo es Linux.

\section*{Routers:}
Básicamente el router es un dispositivo dedicado a la tarea de administrar el tráfico de información que circula por una red de computadoras. Existen dispositivos específicamente diseñados para la función de router, sin embargo, una computadora común puede ser transformada en un router. En la actualidad, un router puede ser usado para compartir internet, a través de cable, ADSL o WiFi con otras computadoras, proveer protección de firewall, controlar la calidad del servicio y otras varias tareas, principalmente en el ámbito de la seguridad. Un router Wireless o WiFi nos provee acceso a la red local y a internet de forma inalámbrica a cualquier dispositivo, ya sea notebook, tablet, impresoras, discos de almacenamiento o smartphones que esté dentro del alcance de la señal.

\end{document}
