\documentclass[a4paper, 11pt, oneside]{article}

\newcommand{\plogo}{\fbox{$\mathcal{PL}$}} 
\usepackage{amsmath}
\usepackage[utf8]{inputenc} 
\usepackage[T1]{fontenc} 
\usepackage{enumitem}
\usepackage{graphicx}
\usepackage{graphicx}
\usepackage{supertabular}
\usepackage[spanish]{babel}
\graphicspath{{Imagenes/}}

\begin{document} 

\begin{titlepage} 

	\centering 
	
	\scshape 
	
	\vspace*{\baselineskip} 
	
	
	
	\rule{\textwidth}{1.6pt}\vspace*{-\baselineskip}\vspace*{2pt} 
	\rule{\textwidth}{0.4pt} 
	
	\vspace{0.75\baselineskip} 
	
	{\LARGE SELinux}	
	\vspace{0.75\baselineskip} 
	
	\rule{\textwidth}{0.4pt}\vspace*{-\baselineskip}\vspace{3.2pt}
	\rule{\textwidth}{1.6pt} 
	
	\vspace{2\baselineskip} 
	

	ADMINISTRACIÓN DE SISTEMAS UNIX/LINUX
	
	\vspace*{3\baselineskip} 
	
	
	
	Alumna:
	
	\vspace{0.5\baselineskip} 
	
	{\scshape\Large Karla Adriana Esquivel Guzmán \\} 
	\vspace{0.5\baselineskip} 
	\vfill
	\includegraphics{unam.jpg}
	
	\textit{UNIVERSIDAD NACIONAL AUTONOMA DE MEXICO} 
	
	\vfill
	
	
	
	
	\vspace{0.3\baselineskip} 
	
	12/Abril/2019 
	
	 

\end{titlepage}
Hoy vimos con el ayudante una introducción a SELinux, hablamos no a profundidad sobre su funcionamiento.
Es una característica de seguridad de Linux que provee una variedad de políticas de seguridad, incluyendo el estilo de acceso a los controles del Departamento de Defensa de Estados Unidos, a través del uso de módulos de Seguridad en el núcleo Linux. No es una distribución de Linux, sino un set de modificaciones pueden ser aplicado a un sistema Tipo-Unix como Linux y BSD.

\section*{Extra}
\textbf{Políticas en SELinux}\\
Las políticas de seguridad son una serie de reglas para autorizar y denegar operaciones. El módulo de seguridad del kernel (SELinux), valida o invalida operaciones en base a políticas de seguridad predefinidas. Cada proceso posee una política de seguridad asociada, que puede ser exclusiva para aquél o heredada del proceso que lo invocó.

