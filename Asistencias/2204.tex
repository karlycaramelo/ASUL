\documentclass[a4paper, 11pt, oneside]{article}

\newcommand{\plogo}{\fbox{$\mathcal{PL}$}} 
\usepackage{amsmath}
\usepackage[utf8]{inputenc} 
\usepackage[T1]{fontenc} 
\usepackage{enumitem}
\usepackage{graphicx}
\usepackage{graphicx}
\usepackage{supertabular}
\usepackage[spanish]{babel}
\graphicspath{{Imagenes/}}

\begin{document} 

\begin{titlepage} 

	\centering 
	
	\scshape 
	
	\vspace*{\baselineskip} 
	
	
	
	\rule{\textwidth}{1.6pt}\vspace*{-\baselineskip}\vspace*{2pt} 
	\rule{\textwidth}{0.4pt} 
	
	\vspace{0.75\baselineskip} 
	
	{\LARGE Redes}	
	\vspace{0.75\baselineskip} 
	
	\rule{\textwidth}{0.4pt}\vspace*{-\baselineskip}\vspace{3.2pt}
	\rule{\textwidth}{1.6pt} 
	
	\vspace{2\baselineskip} 
	

	ADMINISTRACIÓN DE SISTEMAS UNIX/LINUX
	
	\vspace*{3\baselineskip} 
	
	
	
	Alumna:
	
	\vspace{0.5\baselineskip} 
	
	{\scshape\Large Karla Adriana Esquivel Guzmán \\} 
	\vspace{0.5\baselineskip} 
	\vfill
	\includegraphics{unam.jpg}
	
	\textit{UNIVERSIDAD NACIONAL AUTONOMA DE MEXICO} 
	
	\vfill
	
	
	
	
	\vspace{0.3\baselineskip} 
	
	22/Abril/2019 
	
	 

\end{titlepage}
Hoy vimos un poco sobre redes y Linux, hablamos de la diversidad de usuarios y de los diversos usos.
\begin{itemize}
    \item El comando netstat debe su nombre a network statistics o estadísticas de red. Su labor básica es mostrar una lista de las conexiones entrantes y salientes activas en nuestra computadora. Con netstat podremos localizar problemas de red y comprobar si las conexiones se han realizado correctamente o si alguien algún socket o conexión se encuentran parados. Además de la lista por defecto, netstat se puede acompañar de diferentes opciones, como netstat -r para ver la tabla de enrutamiento o netstat -p para ver qué programas están conectados a sockets abiertos.
    \item El comando host nos dice el nombre de una IP o viceversa, así como la información de una dirección DNS.A partir de una IP cualquiera, sabremos cuál es su nombre de dominio, y a partir del nombre de dominio, obtendremos la IP o direcciones IP asociadas. Host funciona tanto con IPv4 como con IPv6.

    \item El comando ping es uno de los más básicos en redes, ya que nos dice si una IP o nombre de dominio funcionan correctamente o está caído. Para ello, ping envía pequeños paquetes a la dirección especificada para comprobar la conexión. Entre sus particularidades, el comando se ejecuta cada pocos segundos de manera automática, sin fin, a no ser que le indiquemos cuántas comprobaciones queremos. Por ejemplo, ping -c 10 para realizar diez comprobaciones.

    \item  Comando Whois en relación a un dominio. Entre otras cosas, obtendremos datos como el dueño de ese dominio, su servidor y URL de referencia, servidores DNS asociados, estado, cuando se creó el dominio, cuándo se actualizó y cuando expirará. Para obtener más información, podemos ejecutar el comando en forma verbose con whois —verbose.

    \item El comando ifconfig es uno de los más versátiles y completos en relación a redes, ya que entre otras cosas nos dice qué dirección IP tenemos asignada, cuál es la dirección MAC de nuestro dispositivo, etc. Ifconfig también sirve para asignar una IP concreta o para configurar varios parámetros de la red, como configurar una máscara con el comando ifconfig netmask. Además, ofrecen información útil como la cantidad de paquetes enviados y recibidos con o sin errores, paquetes descartados…
\end{itemize}

\end{document}
