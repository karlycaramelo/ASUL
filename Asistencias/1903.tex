\documentclass[a4paper, 11pt, oneside]{article}

\newcommand{\plogo}{\fbox{$\mathcal{PL}$}} 
\usepackage{amsmath}
\usepackage[utf8]{inputenc} 
\usepackage[T1]{fontenc} 
\usepackage{enumitem}
\usepackage{graphicx}
\usepackage{graphicx}
\usepackage{supertabular}
\usepackage[spanish]{babel}
\graphicspath{{Imagenes/}}

\begin{document} 

\begin{titlepage} 

	\centering 
	
	\scshape 
	
	\vspace*{\baselineskip} 
	
	
	
	\rule{\textwidth}{1.6pt}\vspace*{-\baselineskip}\vspace*{2pt} 
	\rule{\textwidth}{0.4pt} 
	
	\vspace{0.75\baselineskip} 
	
	{\LARGE Kernel de Windows}	
	\vspace{0.75\baselineskip} 
	
	\rule{\textwidth}{0.4pt}\vspace*{-\baselineskip}\vspace{3.2pt}
	\rule{\textwidth}{1.6pt} 
	
	\vspace{2\baselineskip} 
	

	ADMINISTRACIÓN DE SISTEMAS UNIX/LINUX
	
	\vspace*{3\baselineskip} 
	
	
	
	Alumna:
	
	\vspace{0.5\baselineskip} 
	
	{\scshape\Large Karla Adriana Esquivel Guzmán \\} 
	\vspace{0.5\baselineskip} 
	\vfill
	\includegraphics{unam.jpg}
	
	\textit{UNIVERSIDAD NACIONAL AUTONOMA DE MEXICO} 
	
	\vfill
	
	
	
	
	\vspace{0.3\baselineskip} 
	
	19/Marzo/2019 
	
	 

\end{titlepage}
La principal característica del Kernel de Windows NT es que es bastante modular, y está basada en dos capas principales, la de usuario y la de kernel. El sistema utiliza cada una para diferentes tipos de programa. Por ejemplo, las aplicaciones se ejecutan en el modo usuario, y los componentes principales del sistema operativo en el modo kernel. Mientras, la mayoría de los drivers suelen usar el modo kernel, aunque con excepciones.Es por eso que se refieren a él como Kernel híbrido, pero sobre todo también porque permite tener subsistemas en el espacio del usuario que se comunicaban con el kernel a través de un mecanismo de IPC. Cuando ejecutas una aplicación, esta accede al modo usuario, donde Windows crea un proceso específico para la aplicación. Cada aplicación tiene su dirección virtual privada, ninguna puede alterar los datos que pertenecen a otra y tampoco acceder al espacio virtual del propio sistema operativo. Es por lo tanto el modo que menos privilegios otorga, incluso el acceso al hardware está limitado, y para pedir los servicios del sistema las aplicaciones tienen que recurrir a la API de Windows. El modo núcleo en cambio es ese en el que el código que se ejecuta en él tiene acceso directo a todo el hardware y toda la memoria del equipo. Aquí todo el código comparte un mismo espacio virtual, y puede incluso acceder a los espacios de dirección de todos los procesos del modo usuario. Esto es peligroso, ya que si un driver en el modo kernel toca lo que no debe podría afectar al funcionamiento de todo el sistema operativo. Este modo núcleo está formado por servicios executive, como el controlador de caché, el gestor de comunicación, gestor de E/S, las llamadas de procedimientos locales, o los gestores de energía y memoria entre otros. Estos a su vez están formados por varios módulos que realizan tareas específicas, controladores de núcleo, un núcleo y una Capa de Abstracción del Hardware o HAL.

\section*{Extra}
\textbf{Diferencias entre el Kernel de Linux y Windows}\\
La principal diferencia entre el Kernel de los sistemas operativos Windows y el de Linux está en su filosofía. El desarrollado por el equipo de Linus Torvalds es de código abierto y cualquiera puede cogerlo y modificarlo, algo que le sirve para estar presente en múltiples sistemas operativos o distros GNU/Linux. El de Microsoft en cambio es bastante más cerrado, y está hecho por y para el sistema operativo Windows. En esencia, en Linux cogieron los principios de modularidad de Unix y decidieron abrir el código y las discusiones técnicas. Gracias a ello, Linux ha creado una comunidad meritocrática de desarrolladores, una en la que todos pueden colaborar y en la que cada cambio que se sugiere se debate con dureza para desechar las peores ideas y quedarse con las mejores. También se halaga a quienes consiguen mejorar las funcionalidades más veteranas.

\end{document}
