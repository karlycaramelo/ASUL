\documentclass[a4paper, 11pt, oneside]{article}

\newcommand{\plogo}{\fbox{$\mathcal{PL}$}} 
\usepackage{amsmath}
\usepackage[utf8]{inputenc} 
\usepackage[T1]{fontenc} 
\usepackage{enumitem}
\usepackage{graphicx}
\usepackage{graphicx}
\usepackage{supertabular}
\usepackage[spanish]{babel}
\graphicspath{{Imagenes/}}

\begin{document} 

\begin{titlepage} 

	\centering 
	
	\scshape 
	
	\vspace*{\baselineskip} 
	
	
	
	\rule{\textwidth}{1.6pt}\vspace*{-\baselineskip}\vspace*{2pt} 
	\rule{\textwidth}{0.4pt} 
	
	\vspace{0.75\baselineskip} 
	
	{\LARGE Resumen 05: Sobre Procesamiento de Texto}	
	\vspace{0.75\baselineskip} 
	
	\rule{\textwidth}{0.4pt}\vspace*{-\baselineskip}\vspace{3.2pt}
	\rule{\textwidth}{1.6pt} 
	
	\vspace{2\baselineskip} 
	

	ADMINISTRACIÓN DE SISTEMAS UNIX/LINUX
	
	\vspace*{3\baselineskip} 
	
	
	
	Alumna:
	
	\vspace{0.5\baselineskip} 
	
	{\scshape\Large Karla Adriana Esquivel Guzmán \\} 
	\vspace{0.5\baselineskip} 
	\vfill
	\includegraphics{unam.jpg}
	
	\textit{UNIVERSIDAD NACIONAL AUTONOMA DE MEXICO} 
	
	\vfill
	
	
	
	
	\vspace{0.3\baselineskip} 
	
	01/Febrero/2019 
	
	 

\end{titlepage}
En esta clase se habló sobre procesamiento y visualización de texto, además aprendimos el uso de nuevos comandos para ello. Los comando son los siguientes:

\begin{itemize}
 \item more: Sirve para visualizar un archivo por páginas.
 \item wc: Se utiliza para encontrar el número de ``newlines''.
 \item awk: Sirve para escanear archivos linea por linea.
 \item grep: Utilizando una expresión regular, procesa linea por linea de texto y luego imprime cualquier linea que sea igual al patrón especificado.
 \item cat: Sirve para concatenar cadenas de texto.
 \item sed: Sirve para modificar cada linea en un archivo, se utiliza para reemplazar una parte específica de la linea.
 \item vim: Es un editor de texto.
 \item less: Sirve para mostrar el contenido de un archivo.
 \item nano: Es un editor de texto.
 \item echo: Es utilizado para mostrar una cadena o linea de texto que ha sido pasado como argumento.
 \item cut: Sirve para cortar una sección de cada linea de código.
 \item tr: Automaticamente traduce (sustituye o mapea) un conjunto de caracteres a otro.
 \item tail: Imprime por default las 10 últimas lineas de código de un archivo.
 \item head: Imprime por default las primeras 10 lineas de código de un archivo.
\end{itemize}


\end{document}
