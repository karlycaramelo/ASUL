\documentclass[a4paper, 11pt, oneside]{article}

\newcommand{\plogo}{\fbox{$\mathcal{PL}$}} 
\usepackage{amsmath}
\usepackage[utf8]{inputenc} 
\usepackage[T1]{fontenc} 
\usepackage{enumitem}
\usepackage{graphicx}
\usepackage{graphicx}
\usepackage{supertabular}
\usepackage[spanish]{babel}
\graphicspath{{Imagenes/}}

\begin{document} 

\begin{titlepage} 

	\centering 
	
	\scshape 
	
	\vspace*{\baselineskip} 
	
	
	
	\rule{\textwidth}{1.6pt}\vspace*{-\baselineskip}\vspace*{2pt} 
	\rule{\textwidth}{0.4pt} 
	
	\vspace{0.75\baselineskip} 
	
	{\LARGE Más Sobre Linux y las Diversas distribuciones}	
	\vspace{0.75\baselineskip} 
	
	\rule{\textwidth}{0.4pt}\vspace*{-\baselineskip}\vspace{3.2pt}
	\rule{\textwidth}{1.6pt} 
	
	\vspace{2\baselineskip} 
	

	ADMINISTRACIÓN DE SISTEMAS UNIX/LINUX
	
	\vspace*{3\baselineskip} 
	
	
	
	Alumna:
	
	\vspace{0.5\baselineskip} 
	
	{\scshape\Large Karla Adriana Esquivel Guzmán \\} 
	\vspace{0.5\baselineskip} 
	\vfill
	\includegraphics{unam.jpg}
	
	\textit{UNIVERSIDAD NACIONAL AUTONOMA DE MEXICO} 
	
	\vfill
	
	
	
	
	\vspace{0.3\baselineskip} 
	
	06/Abril/2019 
	
	

\end{titlepage}
Hoy continuamos descargando distribuciones de Linux que aparecen en el Ranking de Distrowatch, pero ahora con una hoja de calculo que subió el profesor a Classroom que tenemos que llenar entre todos los equipos, con los datos ya mencionados en el resumen de la clase anterior.
\section*{Extra}
\textbf{Ventajas de Usar Linux}
\begin{itemize}
    \item Es código abierto.
    \item Puede vivir en maquinas viejas por las distribuciones que exigen pocos recursos.
    \item Es compatible con todos los lenguajes de programación.
    \item Gran variedad de Distribuciones.
    \item Como es un sistema que se podría decir es "comunitario" hay gran variedad de ayuda y soporte por parte de otros usuarios en internet.
\end{itemize}
\end{document}
