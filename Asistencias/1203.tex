\documentclass[a4paper, 11pt, oneside]{article}

\newcommand{\plogo}{\fbox{$\mathcal{PL}$}} 
\usepackage{amsmath}
\usepackage[utf8]{inputenc} 
\usepackage[T1]{fontenc} 
\usepackage{enumitem}
\usepackage{graphicx}
\usepackage{graphicx}
\usepackage{supertabular}
\usepackage[spanish]{babel}
\graphicspath{{Imagenes/}}

\begin{document} 

\begin{titlepage} 

	\centering 
	
	\scshape 
	
	\vspace*{\baselineskip} 
	
	
	
	\rule{\textwidth}{1.6pt}\vspace*{-\baselineskip}\vspace*{2pt} 
	\rule{\textwidth}{0.4pt} 
	
	\vspace{0.75\baselineskip} 
	
	{\LARGE SSH}	
	\vspace{0.75\baselineskip} 
	
	\rule{\textwidth}{0.4pt}\vspace*{-\baselineskip}\vspace{3.2pt}
	\rule{\textwidth}{1.6pt} 
	
	\vspace{2\baselineskip} 
	

	ADMINISTRACIÓN DE SISTEMAS UNIX/LINUX
	
	\vspace*{3\baselineskip} 
	
	
	
	Alumna:
	
	\vspace{0.5\baselineskip} 
	
	{\scshape\Large Karla Adriana Esquivel Guzmán \\} 
	\vspace{0.5\baselineskip} 
	\vfill
	\includegraphics{unam.jpg}
	
	\textit{UNIVERSIDAD NACIONAL AUTONOMA DE MEXICO} 
	
	\vfill
	
	
	
	
	\vspace{0.3\baselineskip} 
	
	12/Marzo/2019 
	
	 

\end{titlepage}
El SSH, Secure Shell, es un protocolo de administración remota a través del cual los usuarios pueden tanto modificar como controlar sus servidores remotos en Internet. Se creó para sustituir a Telnet, un protocolo no cifrado y que por tanto no ofrecía ningún tipo de seguridad a los usuarios. En cambio, SSH hace uso de las técnicas de criptografía más innovadoras con el claro objetivo de que todas las comunicaciones realizadas entre los usuarios y los servidores remotos sean seguras. Dispone de una herramienta que permite autenticar al usuario remoto para posteriormente transferir las entradas desde el cliente al host y, finalmente, realizar la salida de vuelta a los usuarios. Los usuarios de los sistemas operativo Linux y MacOS pueden implantar el protocolo SSH en su servidor remoto de forma muy sencilla a través del terminal. Por supuesto, los usuarios de Windows también pueden hacerlo, aunque el procedimiento es diferente.

\section*{Extra}

En el ámbito de la seguridad informática, hay tres conceptos que resulta interesante conocer:
\begin{itemize}
    \item Cifrado simétrico
    \item Cifrado asimétrico
    \item Hashing.

\end{itemize}

Algunos de los algoritmos de cifrado simétricos que se utilizan con más frecuencia son los siguientes:
\begin{itemize}
    \item DES
    \item 3DES
    \item AES
    \item RC4
\end{itemize}
Los algoritmos 3DES y AES se utilizan de forma habitual por el protocolo IPSEC para establecer las conexiones de servidores virtuales privados. Mientras, el algoritmo RC4 se utiliza para el cifrado de información en tecnologías que hacen uso de las redes inalámbricas.

\end{document}
