\documentclass[a4paper, 11pt, oneside]{article}

\newcommand{\plogo}{\fbox{$\mathcal{PL}$}} 
\usepackage{amsmath}
\usepackage[utf8]{inputenc} 
\usepackage[T1]{fontenc} 
\usepackage{enumitem}
\usepackage{graphicx}
\usepackage{graphicx}
\usepackage{supertabular}
\usepackage[spanish]{babel}
\graphicspath{{Imagenes/}}

\begin{document} 

\begin{titlepage} 

	\centering 
	
	\scshape 
	
	\vspace*{\baselineskip} 
	
	
	
	\rule{\textwidth}{1.6pt}\vspace*{-\baselineskip}\vspace*{2pt} 
	\rule{\textwidth}{0.4pt} 
	
	\vspace{0.75\baselineskip} 
	
	{\LARGE Descarga de Distribuciones, Distrowatch}	
	\vspace{0.75\baselineskip} 
	
	\rule{\textwidth}{0.4pt}\vspace*{-\baselineskip}\vspace{3.2pt}
	\rule{\textwidth}{1.6pt} 
	
	\vspace{2\baselineskip} 
	

	ADMINISTRACIÓN DE SISTEMAS UNIX/LINUX
	
	\vspace*{3\baselineskip} 
	
	
	
	Alumna:
	
	\vspace{0.5\baselineskip} 
	
	{\scshape\Large Karla Adriana Esquivel Guzmán \\} 
	\vspace{0.5\baselineskip} 
	\vfill
	\includegraphics{unam.jpg}
	
	\textit{UNIVERSIDAD NACIONAL AUTONOMA DE MEXICO} 
	
	\vfill
	
	
	
	
	\vspace{0.3\baselineskip} 
	
	05/Abril/2019 
	
	

\end{titlepage}
Esta clase se dejó como practica/tarea por equipo bajar distribuciones de Linux que aparecen en el ranking de distrowatch, se nos asignaron algunas por equipo, lo que se tiene que hacer es utilizar el comando:
\begin{verbatim}
    time wget + [enlace de descarga]
\end{verbatim}
Esto para saber que tiempo nos tomó exactamente bajar cada distribución y poder tomar ciertos datos, como el tiempo de instalación y los recursos que utiliza.

\section*{Extra}
\textbf{Más sobre el comando wget}\\
wget no es interactivo, lo que significa que puede funcionar en segundo plano, mientras que el usuario no ha iniciado sesión, lo que le permite iniciar una recuperación y desconectarse del sistema, lo que le permite a wget terminar el trabajo. Por el contrario, la mayoría de los navegadores web requieren una interacción constante del usuario, lo que dificulta la transferencia de una gran cantidad de datos.
\end{document}
