\documentclass[a4paper, 11pt, oneside]{article}

\newcommand{\plogo}{\fbox{$\mathcal{PL}$}} 
\usepackage{amsmath}
\usepackage[utf8]{inputenc} 
\usepackage[T1]{fontenc} 
\usepackage{enumitem}
\usepackage{graphicx}
\usepackage{graphicx}
\usepackage{supertabular}
\usepackage[spanish]{babel}
\graphicspath{{Imagenes/}}

\begin{document} 

\begin{titlepage} 

	\centering 
	
	\scshape 
	
	\vspace*{\baselineskip} 
	
	
	
	\rule{\textwidth}{1.6pt}\vspace*{-\baselineskip}\vspace*{2pt} 
	\rule{\textwidth}{0.4pt} 
	
	\vspace{0.75\baselineskip} 
	
	{\LARGE Más información sobre las distribuciones de Linux}	
	\vspace{0.75\baselineskip} 
	
	\rule{\textwidth}{0.4pt}\vspace*{-\baselineskip}\vspace{3.2pt}
	\rule{\textwidth}{1.6pt} 
	
	\vspace{2\baselineskip} 
	

	ADMINISTRACIÓN DE SISTEMAS UNIX/LINUX
	
	\vspace*{3\baselineskip} 
	
	
	
	Alumna:
	
	\vspace{0.5\baselineskip} 
	
	{\scshape\Large Karla Adriana Esquivel Guzmán \\} 
	\vspace{0.5\baselineskip} 
	\vfill
	\includegraphics{unam.jpg}
	
	\textit{UNIVERSIDAD NACIONAL AUTONOMA DE MEXICO} 
	
	\vfill
	
	
	
	
	\vspace{0.3\baselineskip} 
	
	09/Abril/2019 
	
	

\end{titlepage}
Hoy continuamos con las descargas y hablamos sobre las distintas distribuciones de linux
La distribución de Linux es un sistema operativo creado a partir de una colección de software, que se basa en el kernel de Linux y, a menudo, en un sistema de gestión de paquetes. Los usuarios de Linux generalmente obtienen su sistema operativo mediante la descarga de una de las distribuciones de Linux, que están disponibles para una amplia variedad de sistemas que van desde dispositivos integrados (por ejemplo, OpenWrt) y computadoras personales (por ejemplo, Linux Mint) hasta potentes supercomputadoras (por ejemplo, , Rocks Cluster Distribution). Una distribución típica de Linux comprende un kernel de Linux, herramientas y bibliotecas de GNU, software adicional, documentación, un sistema de ventanas (el más común es el sistema X Window), un administrador de ventanas y un entorno de escritorio. La mayoría del software incluido es gratuito y el software de código abierto está disponible tanto como archivos binarios compilados como en forma de código fuente, permitiendo modificaciones al software original. Por lo general, las distribuciones de Linux incluyen opcionalmente algún software propietario que puede no estar disponible en forma de código fuente, como las burbujas binarias requeridas para algunos controladores de dispositivos. Una distribución de Linux también se puede describir como una variedad particular de aplicaciones y software de utilidad (varias herramientas y bibliotecas de GNU, por ejemplo), empaquetadas junto con el kernel de Linux de tal manera que sus capacidades satisfagan las necesidades de muchos usuarios. El software generalmente se adapta a la distribución y luego se empaqueta en paquetes de software por parte de los mantenedores de la distribución. Los paquetes de software están disponibles en línea en los llamados repositorios, que son ubicaciones de almacenamiento generalmente distribuidas en todo el mundo. Además de los componentes de pegamento, como los instaladores de distribución (por ejemplo, Debian-Installer y Anaconda) o los sistemas de administración de paquetes, solo hay muy pocos paquetes escritos originalmente por los mantenedores de una distribución de Linux.

\end{document}
