\documentclass[a4paper, 11pt, oneside]{article}

\newcommand{\plogo}{\fbox{$\mathcal{PL}$}} 
\usepackage{amsmath}
\usepackage[utf8]{inputenc} 
\usepackage[T1]{fontenc} 
\usepackage{enumitem}
\usepackage{graphicx}
\usepackage{graphicx}
\usepackage{supertabular}
\usepackage[spanish]{babel}
\graphicspath{{Imagenes/}}

\begin{document} 

\begin{titlepage} 

	\centering 
	
	\scshape 
	
	\vspace*{\baselineskip} 
	
	
	
	\rule{\textwidth}{1.6pt}\vspace*{-\baselineskip}\vspace*{2pt} 
	\rule{\textwidth}{0.4pt} 
	
	\vspace{0.75\baselineskip} 
	
	{\LARGE Más Sobre SELinux}	
	\vspace{0.75\baselineskip} 
	
	\rule{\textwidth}{0.4pt}\vspace*{-\baselineskip}\vspace{3.2pt}
	\rule{\textwidth}{1.6pt} 
	
	\vspace{2\baselineskip} 
	

	ADMINISTRACIÓN DE SISTEMAS UNIX/LINUX
	
	\vspace*{3\baselineskip} 
	
	
	
	Alumna:
	
	\vspace{0.5\baselineskip} 
	
	{\scshape\Large Karla Adriana Esquivel Guzmán \\} 
	\vspace{0.5\baselineskip} 
	\vfill
	\includegraphics{unam.jpg}
	
	\textit{UNIVERSIDAD NACIONAL AUTONOMA DE MEXICO} 
	
	\vfill
	
	
	
	
	\vspace{0.3\baselineskip} 
	
	25/Abril/2019 
	
	 

\end{titlepage}
Hoy hablamos nuevamente sobre SELinux y se mencionaron algunos comandos, que además se dejó de tarea investigar más sobre ellos.
\begin{itemize}
    \item semanage:  cambia el contexto SELinux de los archivos. Cuando se usa la política destinada, los cambios hechos con este comando se agregan al archivo 
    \item chcon: Este comando sirve para cambiar accesos y permisos de los usuarios.
    \item seinfo: Este comando le permite al usuario consultar los componentes de una política de SELinux. Puede analizar un binario o una política de origen utilizando esta herramienta.
\end{itemize}
\end{document}
