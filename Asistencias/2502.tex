\documentclass[a4paper, 11pt, oneside]{article}

\newcommand{\plogo}{\fbox{$\mathcal{PL}$}} 
\usepackage{amsmath}
\usepackage[utf8]{inputenc} 
\usepackage[T1]{fontenc} 
\usepackage{enumitem}
\usepackage{graphicx}
\usepackage{graphicx}
\usepackage{supertabular}
\usepackage[spanish]{babel}
\graphicspath{{Imagenes/}}

\begin{document} 

\begin{titlepage} 

	\centering 
	
	\scshape 
	
	\vspace*{\baselineskip} 
	
	
	
	\rule{\textwidth}{1.6pt}\vspace*{-\baselineskip}\vspace*{2pt} 
	\rule{\textwidth}{0.4pt} 
	
	\vspace{0.75\baselineskip} 
	
	{\LARGE Archivos de configuración y SELinux}	
	\vspace{0.75\baselineskip} 
	
	\rule{\textwidth}{0.4pt}\vspace*{-\baselineskip}\vspace{3.2pt}
	\rule{\textwidth}{1.6pt} 
	
	\vspace{2\baselineskip} 
	

	ADMINISTRACIÓN DE SISTEMAS UNIX/LINUX
	
	\vspace*{3\baselineskip} 
	
	
	
	Alumna:
	
	\vspace{0.5\baselineskip} 
	
	{\scshape\Large Karla Adriana Esquivel Guzmán \\} 
	\vspace{0.5\baselineskip} 
	\vfill
	\includegraphics{unam.jpg}
	
	\textit{UNIVERSIDAD NACIONAL AUTONOMA DE MEXICO} 
	
	\vfill
	
	
	
	
	\vspace{0.3\baselineskip} 
	
	25/Febrero/2019 
	
	 

\end{titlepage}

\section*{grub.cfg}
El archivo de configuración (/boot/grub/grub.conf), usado para crear la lista en la interfaz de menú de GRUB de los sistemas operativos para el arranque, básicamente permite al usuario seleccionar un grupo predefinido de comandos para su ejecución.
El archivo de configuración de la interfaz de menú de GRUB es /boot/grub/grub.conf. Los comandos para configurar las preferencias globales para la interfaz de menú están ubicados al inicio del archivo, seguido de las diferentes estrofas para cada sistema operativo o kernels listados en el menú.

\section*{SELinux:}

\textbf{Security-Enhanced Linux (SELinux)} es un módulo de seguridad para el kernel Linux que proporciona el mecanismo para soportar políticas de seguridad para el control de acceso, incluyendo controles de acceso obligatorios como los del Departamento de Defensa de Estados Unidos. Se trata de un conjunto de modificaciones del núcleo y herramientas de usuario que pueden ser agregadas a diversas distribuciones Linux. Su arquitectura se enfoca en separar las decisiones de las aplicaciones de seguridad de las políticas de seguridad mismas y racionalizar la cantidad de software encargado de las aplicaciones de seguridad.




\end{document}
