\documentclass[a4paper, 11pt, oneside]{article}

\newcommand{\plogo}{\fbox{$\mathcal{PL}$}} 
\usepackage{amsmath}
\usepackage[utf8]{inputenc} 
\usepackage[T1]{fontenc} 
\usepackage{enumitem}
\usepackage{graphicx}
\usepackage{graphicx}
\usepackage{supertabular}
\usepackage[spanish]{babel}
\graphicspath{{Imagenes/}}

\begin{document} 

\begin{titlepage} 

	\centering 
	
	\scshape 
	
	\vspace*{\baselineskip} 
	
	
	
	\rule{\textwidth}{1.6pt}\vspace*{-\baselineskip}\vspace*{2pt} 
	\rule{\textwidth}{0.4pt} 
	
	\vspace{0.75\baselineskip} 
	
	{\LARGE Kernel y "Mini imagenes" mini.iso}	
	\vspace{0.75\baselineskip} 
	
	\rule{\textwidth}{0.4pt}\vspace*{-\baselineskip}\vspace{3.2pt}
	\rule{\textwidth}{1.6pt} 
	
	\vspace{2\baselineskip} 
	

	ADMINISTRACIÓN DE SISTEMAS UNIX/LINUX
	
	\vspace*{3\baselineskip} 
	
	
	
	Alumna:
	
	\vspace{0.5\baselineskip} 
	
	{\scshape\Large Karla Adriana Esquivel Guzmán \\} 
	\vspace{0.5\baselineskip} 
	\vfill
	\includegraphics{unam.jpg}
	
	\textit{UNIVERSIDAD NACIONAL AUTONOMA DE MEXICO} 
	
	\vfill
	
	
	
	
	\vspace{0.3\baselineskip} 
	
	06/Marzo/2019 
	
	 

\end{titlepage}

\section*{Kernel:}
El kernel o núcleo es la parte central del sistema operativo. Se podría decir que el kernel funciona como intermediaro entre el software y el hardware, puesto que es el que recibe las órdenes de los elementos del sistema operativo para enviarlas al procesador u otros elementos del hardware para que se ejecuten. El kernel es la primera capa de software donde tenemos los drivers que controlan todos los componentes del hardware como: pantalla, cámara, bluetooth, memoria, USB, Wi-Fi, Audio, Carga, CPU, etc. 

\section*{mini .iso}
Proporciona solo los paquetes necesarios para la instalación. El ahorro de tiempo de descarga alcanzado con un CD mínimo puede ser significativo, ya que solo se descargan los paquetes actuales, por lo que no es necesario actualizarlos inmediatamente después de la instalación. El CD mínimo utiliza un instalador basado en texto como el CD alternativo, lo que hace que la imagen del CD sea lo más compacta posible.

\section*{Extra ¿Qué es el Bootloader?:}
Bootloader es el nombre para el gestor de arranque del dispositivo. Es el primer programa que se ejecuta en el procesador cuando enciendes un smartphone Android. Se encarga de cargar el Kernel Linux y el sistema operativo Android. El bootloader es una parte fundamental de todas las máquinas que ejecutan un sistema operativo ya sean un smartphone o un ordenador.


\end{document}
