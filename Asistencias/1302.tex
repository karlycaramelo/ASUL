\documentclass[a4paper, 11pt, oneside]{article}

\newcommand{\plogo}{\fbox{$\mathcal{PL}$}} 
\usepackage{amsmath}
\usepackage[utf8]{inputenc} 
\usepackage[T1]{fontenc} 
\usepackage{enumitem}
\usepackage{graphicx}
\usepackage{graphicx}
\usepackage{supertabular}
\usepackage[spanish]{babel}
\graphicspath{{Imagenes/}}

\begin{document} 

\begin{titlepage} 

	\centering 
	
	\scshape 
	
	\vspace*{\baselineskip} 
	
	
	
	\rule{\textwidth}{1.6pt}\vspace*{-\baselineskip}\vspace*{2pt} 
	\rule{\textwidth}{0.4pt} 
	
	\vspace{0.75\baselineskip} 
	
	{\LARGE Resumen 12}	
	\vspace{0.75\baselineskip} 
	
	\rule{\textwidth}{0.4pt}\vspace*{-\baselineskip}\vspace{3.2pt}
	\rule{\textwidth}{1.6pt} 
	
	\vspace{2\baselineskip} 
	

	ADMINISTRACIÓN DE SISTEMAS UNIX/LINUX
	
	\vspace*{3\baselineskip} 
	
	
	
	Alumna:
	
	\vspace{0.5\baselineskip} 
	
	{\scshape\Large Karla Adriana Esquivel Guzmán \\} 
	\vspace{0.5\baselineskip} 
	\vfill
	\includegraphics{unam.jpg}
	
	\textit{UNIVERSIDAD NACIONAL AUTONOMA DE MEXICO} 
	
	\vfill
	
	
	
	
	\vspace{0.3\baselineskip} 
	
	12/Febrero/2019 
	
	 

\end{titlepage}

\section*{Conceptos vistos en clase:}
\begin{itemize}
 \item UEFI (Unified Extensible Firmware Interface): Es un gestor de arranque que proporciona una interfaz que proporciona menús gráficos adicionales y proporciona acceso remoto para la solución de problemas.
 \item EFI (Extensible Firmware Interface): Es un gestor de arranque, que pretende sustituir al BIOS.
 \item Maquina Virtual: Es un emulador de un sistema computacional, que está basado en arquitecturas de computadora y proporciona la funcionalidad de una computadora física.
 \item Virtualización: Se refiere a la acción de crear una versión virtual de algo, incluyendo plataformas virtuales de hardware de computadora, dispositivos de almacenamiento y recursos de red de computadoras.
 \item rootkit: Es una colección de software, usualmente malicioso, que permite a un usuario no autorizado modificar el sistema.
\end{itemize}

\section*{Comandos Utilizados:}
\begin{itemize}
 \item pwd: Imprime la ruta del directorio actual.
 \item df -lh: muestra las estadísticas de espacio en disco del sistema de archivos en formato "legible por humanos", significa que proporciona los detalles en bytes, mega bytes y gigabyte.
 \item fdisk -l /dev/sda: Crea y administra particiones en un disco duro.
 \item file *: se utiliza para determinar el tipo de un archivo.

\end{itemize}

Solo revisamos conceptos ya vistos anteriormente, salvo por rootkit, ese si fue un concepto nuevo.
\end{document}
