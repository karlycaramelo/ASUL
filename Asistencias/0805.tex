\documentclass[a4paper, 11pt, oneside]{article}

\newcommand{\plogo}{\fbox{$\mathcal{PL}$}} 
\usepackage{amsmath}
\usepackage[utf8]{inputenc} 
\usepackage[T1]{fontenc} 
\usepackage{enumitem}
\usepackage{graphicx}
\usepackage{graphicx}
\usepackage{supertabular}
\usepackage[spanish]{babel}
\graphicspath{{Imagenes/}}

\begin{document} 

\begin{titlepage} 

	\centering 
	
	\scshape 
	
	\vspace*{\baselineskip} 
	
	
	
	\rule{\textwidth}{1.6pt}\vspace*{-\baselineskip}\vspace*{2pt} 
	\rule{\textwidth}{0.4pt} 
	
	\vspace{0.75\baselineskip} 
	
	{\LARGE Lynx, Pine y Correos}	
	\vspace{0.75\baselineskip} 
	
	\rule{\textwidth}{0.4pt}\vspace*{-\baselineskip}\vspace{3.2pt}
	\rule{\textwidth}{1.6pt} 
	
	\vspace{2\baselineskip} 
	

	ADMINISTRACIÓN DE SISTEMAS UNIX/LINUX
	
	\vspace*{3\baselineskip} 
	
	
	
	Alumna:
	
	\vspace{0.5\baselineskip} 
	
	{\scshape\Large Karla Adriana Esquivel Guzmán \\} 
	\vspace{0.5\baselineskip} 
	\vfill
	\includegraphics{unam.jpg}
	
	\textit{UNIVERSIDAD NACIONAL AUTONOMA DE MEXICO} 
	
	\vfill
	
	
	
	
	\vspace{0.3\baselineskip} 
	
	08/Mayo/2019 
	
	

\end{titlepage}
Hoy hablamos sobre Lynx y lo instalamos, también conocimos el comando pine. 
\begin{itemize}
    \item Lynx es un navegador web que a diferencia de los mas populares este se usa mediante una terminal y la navegación es mediante el modo texto. Lynx puede resultar ser una herramienta bastante atractiva para los amantes de la terminal e incluso para las personas que les gusta maximizar la optimización del uso de los recursos del sistema.
    \item pine es una herramienta de manejo de mensajes orientada a la pantalla. En su configuración predeterminada, pine ofrece un conjunto de funciones limitadas intencionalmente orientadas al usuario novato, pero también tiene una lista creciente de funciones opcionales de usuario avanzado y de preferencia personal.
\end{itemize}
Al final instalamos Fedora Server.
\end{document}
