\documentclass[a4paper, 11pt, oneside]{article}

\newcommand{\plogo}{\fbox{$\mathcal{PL}$}} 
\usepackage{amsmath}
\usepackage[utf8]{inputenc} 
\usepackage[T1]{fontenc} 
\usepackage{enumitem}
\usepackage{graphicx}
\usepackage{graphicx}
\usepackage{supertabular}
\usepackage[spanish]{babel}
\graphicspath{{Imagenes/}}

\begin{document} 

\begin{titlepage} 

	\centering 
	
	\scshape 
	
	\vspace*{\baselineskip} 
	
	
	
	\rule{\textwidth}{1.6pt}\vspace*{-\baselineskip}\vspace*{2pt} 
	\rule{\textwidth}{0.4pt} 
	
	\vspace{0.75\baselineskip} 
	
	{\LARGE Resumen 06: Páginas del Manual}	
	\vspace{0.75\baselineskip} 
	
	\rule{\textwidth}{0.4pt}\vspace*{-\baselineskip}\vspace{3.2pt}
	\rule{\textwidth}{1.6pt} 
	
	\vspace{2\baselineskip} 
	

	ADMINISTRACIÓN DE SISTEMAS UNIX/LINUX
	
	\vspace*{3\baselineskip} 
	
	
	
	Alumna:
	
	\vspace{0.5\baselineskip} 
	
	{\scshape\Large Karla Adriana Esquivel Guzmán \\} 
	\vspace{0.5\baselineskip} 
	\vfill
	\includegraphics{unam.jpg}
	
	\textit{UNIVERSIDAD NACIONAL AUTONOMA DE MEXICO} 
	
	\vfill
	
	
	
	
	\vspace{0.3\baselineskip} 
	
	05/Febrero/2019 
	
	 

\end{titlepage}

En esta clase vimos como por medio de \textbf{groff} (es un sistema de composición tipográfica, software de texto plano) se le da como entrada un texto plano y genera una salida una página del manual. Groff permite especificar los preprocesadores por opciones de linea de comandos y ejecuta automáticamente el postprocesador que es apropiado para el dispositivo seleccionado.\\
El programa \textbf{grog} se puede usar para adivinar la linea de comandos de groff correcta para formatear un archivo. El programa \textbf{groffer} es un visor general para archivos groff y páginas de manual. En los siguientes directorios pueden guardarse los archivos de texto plano.
\section*{Directorios:}
\begin{itemize}
 \item /usr/share/man/man<número> (número puede ser entre 1-8, este depende de la categoría que quieras utilizar)
 \item /usr/local/man/man<número> (número puede ser entre 1-8, este depende de la categoría que quieras utilizar)
\end{itemize}


\end{document}
