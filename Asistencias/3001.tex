\documentclass[a4paper, 11pt, oneside]{article}

\newcommand{\plogo}{\fbox{$\mathcal{PL}$}} 
\usepackage{amsmath}
\usepackage[utf8]{inputenc} 
\usepackage[T1]{fontenc} 
\usepackage{enumitem}
\usepackage{graphicx}
\usepackage{graphicx}
\usepackage{supertabular}
\usepackage[spanish]{babel}
\graphicspath{{Imagenes/}}

\begin{document} 

\begin{titlepage} 

	\centering 
	
	\scshape 
	
	\vspace*{\baselineskip} 
	
	
	
	\rule{\textwidth}{1.6pt}\vspace*{-\baselineskip}\vspace*{2pt} 
	\rule{\textwidth}{0.4pt} 
	
	\vspace{0.75\baselineskip} 
	
	{\LARGE Resumen 03: Primera Clase}	
	\vspace{0.75\baselineskip} 
	
	\rule{\textwidth}{0.4pt}\vspace*{-\baselineskip}\vspace{3.2pt}
	\rule{\textwidth}{1.6pt} 
	
	\vspace{2\baselineskip} 
	

	ADMINISTRACIÓN DE SISTEMAS UNIX/LINUX
	
	\vspace*{3\baselineskip} 
	
	
	
	Alumna:
	
	\vspace{0.5\baselineskip} 
	
	{\scshape\Large Karla Adriana Esquivel Guzmán \\} 
	\vspace{0.5\baselineskip} 
	\vfill
	\includegraphics{unam.jpg}
	
	\textit{UNIVERSIDAD NACIONAL AUTONOMA DE MEXICO} 
	
	\vfill
	
	
	
	
	\vspace{0.3\baselineskip} 
	
	30/Enero/2019 
	
	 

\end{titlepage}

Nuevamente se retomó el concepto de \textbf{variable de entorno}, en el resumen anterior se define el significado de la misma. Hay varios Shell que han sido utilizados por varios compañeros del curso y se mencionaron en clase:
\begin{itemize}
 \item zsh
 \item cron
 \item csh
 \item powerShell
\end{itemize}
Un Shell es un intérprete es decir es un programa que va leyendo comandos y ejecutándolos, simplemente es una interfaz que nos permite acceder a los servicios del sistema operativo.\\
Durante la clase el profesor nos preguntó a todos los alumnos ¿qué distribuciones de Linux habíamos utilizado? a continuación voy a enlistarlas.
\begin{itemize}
 \item Ubuntu
 \item Fedora
 \item Arch
 \item Gentoo
 \item AntiX
 \item Mint
 \item Puppy (utilizado para computadoras con pocos recursos)
\end{itemize}

Además de ello se mencionaron algunos procolos como \textbf{DHCP}(Dinamic HOST Configuration Protocol) el cual asigna dinámicamente una dirección IP y otros parámetros de configuración de red a cada dispositivo y \textbf{LDAP} (Lighweight Directory Access Protocol) este permite el acceso a un servicio de directorio ordenado y distribuido para buscar diversa información en un entorno de red. No vimos conceptos a profundidad.



\end{document}
