\documentclass[a4paper, 11pt, oneside]{article}

\newcommand{\plogo}{\fbox{$\mathcal{PL}$}} 
\usepackage{amsmath}
\usepackage[utf8]{inputenc} 
\usepackage[T1]{fontenc} 
\usepackage{enumitem}
\usepackage{graphicx}
\usepackage{graphicx}
\usepackage{supertabular}
\usepackage[spanish]{babel}
\graphicspath{{Imagenes/}}

\begin{document} 

\begin{titlepage} 

	\centering 
	
	\scshape 
	
	\vspace*{\baselineskip} 
	
	
	
	\rule{\textwidth}{1.6pt}\vspace*{-\baselineskip}\vspace*{2pt} 
	\rule{\textwidth}{0.4pt} 
	
	\vspace{0.75\baselineskip} 
	
	{\LARGE Modificación De Modulos y Paquetes}	
	\vspace{0.75\baselineskip} 
	
	\rule{\textwidth}{0.4pt}\vspace*{-\baselineskip}\vspace{3.2pt}
	\rule{\textwidth}{1.6pt} 
	
	\vspace{2\baselineskip} 
	

	ADMINISTRACIÓN DE SISTEMAS UNIX/LINUX
	
	\vspace*{3\baselineskip} 
	
	
	
	Alumna:
	
	\vspace{0.5\baselineskip} 
	
	{\scshape\Large Karla Adriana Esquivel Guzmán \\} 
	\vspace{0.5\baselineskip} 
	\vfill
	\includegraphics{unam.jpg}
	
	\textit{UNIVERSIDAD NACIONAL AUTONOMA DE MEXICO} 
	
	\vfill
	
	
	
	
	\vspace{0.3\baselineskip} 
	
	27/Febrero/2019 
	
\end{titlepage}
\section*{¿Qué es un módulo?:}
El kernel de Linux es modular porque permite insertar y eliminar código bajo demanda con el fin de añadir o quitar una funcionalidad. Algunas de las funcionalidades que podemos añadir al Kernel mediante los módulos del kernel son las siguientes:

\begin{enumerate}
    \item Usar los drivers privativos de nuestra tarjeta gráfica.
    \item Registrar las temperaturas de componentes de nuestro ordenador.
    \item Que los ventiladores de nuestro ordenador sean gestionados por el software creado por el fabricante de nuestro ordenador.
    \item Hacer funcionar nuestro tarjeta de red wifi.
\end{enumerate}
Para ver los módulos del kernel basta aplicar el siguiente comando:\\
\textbf{ls -R /lib/modules/\$(uname -r)}.
Para saber los módulos utilizados basta con escribir el comando: \textbf{lsmod}.

\section*{Paquetes en Linux:}
Un paquete es un conjunto de ficheros relacionados con una aplicación, que contiene los objetos ejecutables, los archivos de configuración, información acerca del uso e instalación de la aplicación, todo ello agrupado en un mismo contenedor.  Encontramos los binarios y los que son el código fuente.

\begin{itemize}
    \item Paquetes binarios: Contienen código maquina, y no código fuente, por lo que cada tipo de arquitectura necesita su propio paquete. Encontramos estos tipos de paquetes binarios:
\begin{itemize}
    \item RPM Estos paquetes son utilizados por distribuciones Red Hat, Suse, Mandrake, Conectiva, Caldera, etc. 

    \item DEB Estos paquetes son utilizados por distribuciones como Debian, y las basadas en ella, como Ubuntu. La utilidad para manejar este tipo de paquetes son apt  y dpkg.

    \item TGZ Son utilizados por la distribución Linux Slackware.
 
\end{itemize}

\item Paquetes de código fuente: Contienen el código fuente del programa, estos vienen con los archivos necesarios para compilar e instalar el programa manualmente. Suelen presentarse en formato .tar.gz o tar.bz2 (osea compactado con tar y comprimido con gzip o bzip). Lo normal es que cada aplicación tenga la informacion en el fichero README o INSTALL de  como instalarlo.
 
\end{itemize}



\end{document}
