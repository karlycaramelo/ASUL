\documentclass[a4paper, 11pt, oneside]{article}

\newcommand{\plogo}{\fbox{$\mathcal{PL}$}} 
\usepackage{amsmath}
\usepackage[utf8]{inputenc} 
\usepackage[T1]{fontenc} 
\usepackage{enumitem}
\usepackage{graphicx}
\usepackage{graphicx}
\usepackage{supertabular}
\usepackage[spanish]{babel}
\graphicspath{{Imagenes/}}

\begin{document} 

\begin{titlepage} 

	\centering 
	
	\scshape 
	
	\vspace*{\baselineskip} 
	
	
	
	\rule{\textwidth}{1.6pt}\vspace*{-\baselineskip}\vspace*{2pt} 
	\rule{\textwidth}{0.4pt} 
	
	\vspace{0.75\baselineskip} 
	
	{\LARGE IPTables}	
	\vspace{0.75\baselineskip} 
	
	\rule{\textwidth}{0.4pt}\vspace*{-\baselineskip}\vspace{3.2pt}
	\rule{\textwidth}{1.6pt} 
	
	\vspace{2\baselineskip} 
	

	ADMINISTRACIÓN DE SISTEMAS UNIX/LINUX
	
	\vspace*{3\baselineskip} 
	
	
	
	Alumna:
	
	\vspace{0.5\baselineskip} 
	
	{\scshape\Large Karla Adriana Esquivel Guzmán \\} 
	\vspace{0.5\baselineskip} 
	\vfill
	\includegraphics{unam.jpg}
	
	\textit{UNIVERSIDAD NACIONAL AUTONOMA DE MEXICO} 
	
	\vfill
	
	
	
	
	\vspace{0.3\baselineskip} 
	
	22/Marzo/2019 
	
	 

\end{titlepage}
Para Iniciar/Parar/Reiniciar Iptables debemos ejecutar estos comandos:
\begin{itemize}
    \item sudo service iptables start
    \item sudo service iptables stop
    \item sudo service iptables restart\\
    También existen banderas dentro de IpTables
    \begin{itemize}
        \item -A –append $\rightarrow$ agrega una regla a una cadena.
        \item -D –delete $\rightarrow$ borra una regla de una cadena especificada.
        \item -R –replace $\rightarrow$ reemplaza una regla.
        \item -I –insert $\rightarrow$ inserta una regla en lugar de una cadena.
        \item -L –list $\rightarrow$ muestra las reglas que le pasamos como argumento.
        \item -F –flush $\rightarrow$ borra todas las reglas de una cadena.
        \item -Z –zero $\rightarrow$ pone a cero todos los contadores de una cadena.
        \item -N –new-chain $\rightarrow$ permite al usuario crear su propia cadena.
        \item -X –delete-chain $\rightarrow$ borra la cadena especificada.
        \item -P –policy $\rightarrow$ explica al kernel qué hacer con los paquetes que no coincidan con ninguna regla.
        \item -E –rename-chain $\rightarrow$ cambia el orden de una cadena.

    \end{itemize}
\end{itemize}
\end{document}
