\documentclass[a4paper, 11pt, oneside]{article}

\newcommand{\plogo}{\fbox{$\mathcal{PL}$}} 
\usepackage{amsmath}
\usepackage[utf8]{inputenc} 
\usepackage[T1]{fontenc} 
\usepackage{enumitem}
\usepackage{graphicx}
\usepackage{graphicx}
\usepackage{supertabular}
\usepackage{hyperref}
\usepackage[spanish]{babel}
\graphicspath{{Imagenes/}}

\begin{document} 

\begin{titlepage} 

	\centering 
	
	\scshape 
	
	\vspace*{\baselineskip} 
	
	
	
	\rule{\textwidth}{1.6pt}\vspace*{-\baselineskip}\vspace*{2pt} 
	\rule{\textwidth}{0.4pt} 
	
	\vspace{0.75\baselineskip} 
	
	{\LARGE Reseña sobre Just For Fun}	
	\vspace{0.75\baselineskip} 
	
	\rule{\textwidth}{0.4pt}\vspace*{-\baselineskip}\vspace{3.2pt}
	\rule{\textwidth}{1.6pt} 
	
	\vspace{2\baselineskip} 
	

	ADMINISTRACIÓN DE SISTEMAS UNIX/LINUX
	
	\vspace*{1\baselineskip} 
	
	\vfill
	
	
	Alumna:
	
	\vspace{0.2\baselineskip} 
	
	{\scshape\Large Karla Adriana Esquivel Guzmán}
	\vspace{0.5\baselineskip} 
	\vfill
	\includegraphics{unam.jpg}
	
	\textit{UNIVERSIDAD NACIONAL AUTONOMA DE MEXICO} 
	
	
	
	
	
	\vspace{0.3\baselineskip} 
	
	01/Abril/2019 
	
	 

\end{titlepage}

El libro de \textbf{Just For Fun}es una autobiografía, la idea principal se centra en la motivación, para la creación de Linux, así como diversas vivencias de su creador Linus Torvals.

Linus nació en Finlandia, después de que su abuelo lo introdujera al mundo de las computadoras, su vida dio un cambio radical.

Cuando era estudiante, era "mal visto" ser un \textbf{Nerd}, sin embargo el se consideraba ésta clase de persona, era bueno para las asignaturas que más dificultad presentaban para la mayoría de personas, como matemáticas o física, y como un buen cliché, no tenía suerte con las chicas, no era popular y tampoco bueno en deportes.

Después de un intento por entrar al ejercito, algunos proyectos y la lectura de un libro de Tanenbaum, comenzó a experimentar con el sistema operativo Minix, que es una copia de UNIX (basado en el estándar de POSIX). Linus quería mejorar dicho sistema, se tomó varios meses escribiendo diversas llamadas al sistema, durante ésta época de su vida, se nota como se "clava" en éste proyecto y su vida se basaba en codificar, comer y dormir, incluso estaba aislado de tanto trabajar. Aun trabajando en el desarrollo de Linux, Linus conoció a Tove, quién en un futuro se casó con él, me gusta que además de mencionar a detalle, todo sobre su trabajo, también habla sobre su vida personal, en paralelo, y muchas de las cosas ocurridas durante el desarrollo de Linux, fueron afortunadas coincidencias.


En general el libro es entretenido y te adentra en la vivencia de Linus, y la verdad es que es emocionante ver que un sistema operativo que comenzó a ser desarrollado como un proyecto personal, es ahora uno de los sistemas operativos más importantes de nuestra época.
\end{document}