\documentclass[a4paper, 11pt, oneside]{article}

\newcommand{\plogo}{\fbox{$\mathcal{PL}$}} 
\usepackage{amsmath}
\usepackage[utf8]{inputenc} 
\usepackage[T1]{fontenc} 
\usepackage{enumitem}
\usepackage{graphicx}
\usepackage{graphicx}
\usepackage{supertabular}
\usepackage[spanish]{babel}
\graphicspath{{Imagenes/}}

\begin{document} 

\begin{titlepage} 

	\centering 
	
	\scshape 
	
	\vspace*{\baselineskip} 
	
	
	
	\rule{\textwidth}{1.6pt}\vspace*{-\baselineskip}\vspace*{2pt} 
	\rule{\textwidth}{0.4pt} 
	
	\vspace{0.75\baselineskip} 
	
	{\LARGE Mitos y Realidades sobre Linux}	
	\vspace{0.75\baselineskip} 
	
	\rule{\textwidth}{0.4pt}\vspace*{-\baselineskip}\vspace{3.2pt}
	\rule{\textwidth}{1.6pt} 
	
	\vspace{2\baselineskip} 
	

	ADMINISTRACIÓN DE SISTEMAS UNIX/LINUX
	
	\vspace*{3\baselineskip} 
	
	
	
	Alumna:
	
	\vspace{0.5\baselineskip} 
	
	{\scshape\Large Karla Adriana Esquivel Guzmán \\} 
	\vspace{0.5\baselineskip} 
	\vfill
	\includegraphics{unam.jpg}
	
	\textit{UNIVERSIDAD NACIONAL AUTONOMA DE MEXICO} 
	
	\vfill
	
	
	
	
	\vspace{0.3\baselineskip} 
	
	26/Mayo/2019 
	
	

\end{titlepage}
Como usuarios, cuando aún no tenemos mucha experiencia utilizando Linux, en algunas ocasiones ni siquiera nos damos el tiempo para conocer el Sistema Operativo por nosotros mismos y nos dejamos llevar por prejuicios o mitos sobre Linux, a continuación enunciaré los que yo escuché más frecuentemente antes de ser una usuaria frecuente de Linux.
\begin{itemize}
    \item Linux es díficil de utilizar, no se le entiende nada.
    \item Linux es solo para gente muy experimentada en el area de redes.
    \item ¿Para qué aprendes a utilizar Linux?, si está bien feo y ni siquiera hay programas que Windows ya tiene.
    \item Windows es un mejor sistema operativo para programadores que Linux, además eso nadie lo ocupa actualmente.
    \item Linux es gratuito y por lo tanto no es mejor que ningún sistema operativo comercial.
\end{itemize}
Sin embargo durante el proceso en que he ido adentrandome más en Linux y conociendo personalmente sus ventajas me he dado cuenta de lo equivocada que está la gente en general, que incluso las personas de area de Sistemas están equivocados al hacerle tan mala reputación a Linux a continuación enuncio las Realidades sobre Linux además de desmentir los mitos que ya mencioné con anterioridad.
\begin{itemize}
    \item Linux es tan difícil de aprender a utilizar como cualquier Software que intentes utilizar por primera vez, es decir dependerá más de le perseverancia de la persona y de las ganas que tenga de aprender a utilizarlo, en lo personal lo considero muy sencillo de utilizar.
    \item Linux no es solo para usuarios experimentados ni adentrados en el area de Redes o Sistema, la realidad es que nos han vendido la idea de que solo existen Windows y MacOS, por ello muchas personas no familiarizadas con las areas ya mencionadas podrían ni siquiera conocerlo, también por obra del capitalismo es que estos sistemas operativos predominan como los más utilizados, pues también la mayoría de programas de uso general como videojuegos, aplicaciones de música populares, etcétera, solo las sacan en versión para Windows o MacOS y muchas veces no se tiene una versión para Linux. Pero utilizar Linux es muy sencillo para cualquier usuario que deseé cambiarse de Sistema Operativo y lo que yo considero una gran ventaja es que ¡es Gratis!.
    \item Aprender a utilizar Linux nos da una gran ventaja sobre los demás programadores, pues Linux ofrece funciones de seguridad y de manejo de sistema de archivos que son una maravilla y facilitan mucho el trabajo, además todos los lenguajes de programación son compatibles con Linux.
    \item Linux es Gratuito, libre y comunitario, lo cual hace que haya más soporte por parte de la comunidad y entre muchas personas al rededor del mundo puedan solucionarse en común problemas que lleguen a surgir con alguna distribución de Linux.
    \item Es súper importante también mencionar que Linux cuenta con cientos de distribuciones que pueden servir para cualquier tipo de usuario, desde las más amigables con el usuario, como lo es ubuntu hasta distribuciones que puedes configurar desde 0 como Arch o Gentoo, Linux simplemente es una maravilla.
    \item Algo que no dejaré de lado es que hay infinidad de programas Open source para Linux, y programas gratuitos que nos permitiran cumplir con tareas muy básicas o muy complicadas, tratandose del area de sistemas.
    \begin{center}
        \includegraphics[scale=0.10]{Linux.png}
    \end{center}
\end{itemize}
\end{document}