\documentclass[a4paper, 11pt, oneside]{article}

\newcommand{\plogo}{\fbox{$\mathcal{PL}$}} 
\usepackage{amsmath}
\usepackage[utf8]{inputenc} 
\usepackage[T1]{fontenc} 
\usepackage{enumitem}
\usepackage{graphicx}
\usepackage{graphicx}
\usepackage{supertabular}
\usepackage[spanish]{babel}
\usepackage{hyperref}
\graphicspath{{Imagenes/}}

\begin{document} 

\begin{titlepage} 

	\centering 
	
	\scshape 
	
	\vspace*{\baselineskip} 
	
	
	
	\rule{\textwidth}{1.6pt}\vspace*{-\baselineskip}\vspace*{2pt} 
	\rule{\textwidth}{0.4pt} 
	
	\vspace{0.75\baselineskip} 
	
	{\LARGE Correos entre Andy Tanenbaum y Linus Torvalds}	
	\vspace{0.75\baselineskip} 
	
	\rule{\textwidth}{0.4pt}\vspace*{-\baselineskip}\vspace{3.2pt}
	\rule{\textwidth}{1.6pt} 
	
	\vspace{2\baselineskip} 
	

	ADMINISTRACIÓN DE SISTEMAS UNIX/LINUX
	
	\vspace*{3\baselineskip} 
	
	
	
	Alumnos:
	
	\vspace{0.5\baselineskip} 
	
	{\scshape\Large Karla Adriana Esquivel Guzmán url{https://github.com/karlycaramelo} \\}
	\vspace{0.5\baselineskip} 
	\vfill
	\includegraphics[scale=0.65]{unam.jpg}
	
	\textit{UNIVERSIDAD NACIONAL AUTONOMA DE MEXICO} 
	
	\vfill
	
	
	
	
	\vspace{0.3\baselineskip} 
	
    12/Mayo/2019 
	
	 

\end{titlepage}

\section*{Comentario}

Me parece muy prepotente Andy Tanenbaum en sus correos, de hecho se nota el menosprecio hacía Linux, pero me agrada la manera en que Linus Torvalds le responde sin miedo, me parece que le responde medio "passive-aggressive", pero concuerdo en que es importante defender el trabajo que uno hace, me hubiese gustado que en los correos de Tanenbaum a Torvalds, en lugar de críticas destructivas hubiese habido más retroalimentación por parte de Tanenbaum pues él ya era un investigador experimentado, pero se le nota que es muy hermético, creo que parte de esos correos me recuerda a lo que vivimos día a día en la academia, me parece que los Profesores e investigadores deberían estar más comprometidos a apoyar a los estudiantes que están interesadas en trabajar en el area que ellos conocen, muchas veces en lugar de dar retroalimentación a los estudiantes los menosprecian por su juventud e inexperiencia, pero justo por eso piden ayuda y lo que reciben es un mal trato. Pero volviendo a lo anterior, yo he tenido la experiencia de trabajar con \textbf{Minix} es un sistema operativo, que pide pocos recursos, es muy amigable para trabajos y practicas escolares, ciertamente es un sistema operativo pequeño que está hecho con fines académicos, Linux por otro lado es un "monstruo" en comparación con Minix pues Linux si fue diseñado como una herramienta de trabajo, me alegra que Linus Torvalds, no se haya detenido en su creación por críticas negativas, pues nos ha brindado un sistema operativo gratuito y de código abierto.
\end{document}